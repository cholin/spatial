\documentclass[english]{lni}
\let\ifpdf\relax
\IfFileExists{latin1.sty}{\usepackage{latin1}}{\usepackage{isolatin1}}

\usepackage{ifpdf}
\usepackage{graphicx}
\usepackage{hyperref}

\author{
	{Adam Furmańczuk, Nico von Geyso} \\
	\\
	Databases and Information Systems\\
	Department of Computer Sciences\\
	Freie Universitaet Berlin\\
	Takustrasse \\
	14195 Berlin  \\
	adam.furmanczuk@fu-berlin.de \\
	nico.geyso@FU-Berlin.de
}
\title{Spatial Database Project}
\begin{document}
\maketitle

\begin{abstract}
In this spaital databae project historical and forecast data is collected and overlayed on a map. 
\end{abstract}

\section{Introduction}
The task in spatial database project is to visualize historical and current weather measurements and forecasts.
Part of the assignment is to collect forecast and historical weather data and overlay that data with OSM or another map provider.

\section{Data Sources}

\subsection{Historical Data}
Main datasource for historical data is {\textit{Deutsche Wetterdienst}}. Data for temperature, air pressure and rainfall is collected from 503 weather stations in Germany. The data is available through public ftp server  \hyperref{ftp://ftp.dwd.de/pub/CDC/observations_germany/}. It contains data for the past (several month up to years) until today.

\subsection{Forecast Data}

Forecast data is provieded by National Oceanic and Atmospheric Administration (NOAA) Global Forecast System,
It provides a global weather forecast model. Data is freely available for current and past forecasts.
The data can be accessed via ftp server or by perl webserver script. Output format is grib2 (raster).

\section{Implementation}

\subsection{Implementation overview}
The implementation is based on python and postgres database with postgis extension. Client communication is done over REST api.
The python libraries include:
\begin{itemize}
		\item flask web framework
		\item geoalchemy
		\item numpy
		\item shapely
\end{itemize}

\subsection{Installation}

The source code along with the installation instruction is available on github \url{https://github.com/cholin/fuberlin_spatial_db_project/}.

Basically, in order to run the project on the local (linux) machine the following steps need to be taken:

\begin{itemize}
\item make sure python2.7, pip,  postgres9.x, postgis2.x are installed
\item setup database for postgres:
\begin{itemize}
\item sudo su postgres
\item createuser -D -P -s -R <username>
\item createdb -O <username> <database>
\end{itemize}
\item git clone git@github.com:cholin/spatial.git
\item cd spatial
\item pip install -r requirements.txt
\item cat config.cfg.dist > config.cfg
\item vi config.cfg \# enter user credentials and database name
\item python2 manage.py resetdb \# delete and setup database with postgis extension
\item python2 manage.py import\_weather  \# import historical weather from DWD
\item python2 manage.py import\_forecast YYYY-mm-dd \# import forecast since given date  
\item python2 manage.py runserver
\end{itemize}

The spatial database project is now available on \url{http://localhost:5000}.

\section{Issues}
\subsection{Weather data provided as points}
Weather data from DWD is provided as a collection of geographic points. In order to produce a nice overlay map, those points need to be transformed to adjacent polygons. We use a Vernoi Alogirthm, to transform the points into cells.

\subsection{Importing weather data}
For the datasources nearly no or bad documentation is available. This problem is most severe on the NOAA plattform. where variables are not documented and one need to try-and-error on which raster-band are the correct values.

Another problem was that some data is inconsistent or missing. We chose to leave are blank, where no data is available.

As only few weather data sources are freely accessible. It is an time consuming issue to find and import the data at all.
Another point is, that raster data from NOAA is provided with a 0.5 degree resolution. We use a filter technique for a nicer visualization.

\subsection{Learning the libraries}

The last issue encountered was get familiar with libraries.

\section{Outlook}

This project can be further extended for calculations, datasets and visualisation.
As for calculations, based on the available data a new forecast system can be introduced. 
Furhter development effort can be put to include more datasets. Currently, we use only DWD and NOAA data sources. include and integrate more forecast sources.
Finally, more work can be put into visualization, in order to improve usability. 

\section{Summary}
In this project we learned to account for vector and raster data in postgis. We also learned how to visualize spatial data using front-end library leaflet in connection with the database extension postgis.

\bibliography{lnitemplate}

\end{document}
