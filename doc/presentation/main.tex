\documentclass[ucs,9pt]{beamer}

% Copyright 2004 by Till Tantau <tantau@users.sourceforge.net>.
%
% In principle, this file can be redistributed and/or modified under
% the terms of the GNU Public License, version 2.
%
% However, this file is supposed to be a template to be modified
% for your own needs. For this reason, if you use this file as a
% template and not specifically distribute it as part of a another
% package/program, I grant the extra permission to freely copy and
% modify this file as you see fit and even to delete this copyright
% notice.
%
% Modified by Tobias G. Pfeiffer <tobias.pfeiffer@math.fu-berlin.de>
% to show usage of some features specific to the FU Berlin template.

% remove this line and the "ucs" option to the documentclass when your editor is not utf8-capable
\usepackage[utf8x]{inputenc}    % to make utf-8 input possible
\usepackage[english]{babel}     % hyphenation etc., alternatively use 'german' as parameter
\usepackage{graphicx}
\usepackage{subcaption}
\captionsetup{compatibility=false}


\include{fu-beamer-template}  % THIS is the line that includes the FU template!

\usepackage{arev,t1enc} % looks nicer than the standard sans-serif font
% if you experience problems, comment out the line above and change
% the documentclass option "9pt" to "10pt"

% image to be shown on the title page (without file extension, should be pdf or png)
\titleimage{fu_500}

\title[Short Paper Title] % (optional, use only with long paper titles)
{Weather forecast visualization}

\subtitle
{Spatial DB project}

\author[Author, Another] % (optional, use only with lots of authors)
{Adam Furmańczuk \and Nico von Geyso}
% - Give the names in the same order as the appear in the paper.

\institute[FU Berlin] % (optional, but mostly needed)
{Freie Universität Berlin}
% - Keep it simple, no one is interested in your street address.

\date[CFP 2003] % (optional, should be abbreviation of conference name)
{February 2015}
% - Either use conference name or its abbreviation.
% - Not really informative to the audience, more for people (including
%   yourself) who are reading the slides online

\subject{Theoretical Computer Science}
% This is only inserted into the PDF information catalog. Can be left
% out.

% you can redefine the text shown in the footline. use a combination of
% \insertshortauthor, \insertshortinstitute, \insertshorttitle, \insertshortdate, ...
\renewcommand{\footlinetext}{\insertshortinstitute, \insertshorttitle, \insertshortdate}

% Headline
\newcommand\headline[1]{%
  \par\bigskip
  {\Large\bfseries#1}\par\smallskip}

% Delete this, if you do not want the table of contents to pop up at
% the beginning of each subsection:
\AtBeginSubsection[]
{
  \begin{frame}<beamer>{Outline}
    \tableofcontents[currentsection,currentsubsection]
  \end{frame}
}

\begin{document}

\begin{frame}[plain]
  \titlepage
\end{frame}

\begin{frame}{Outline}
  \tableofcontents
  % You might wish to add the option [pausesections]
\end{frame}

\section{Introduction}

\begin{frame}{Topic}
  \begin{block}{Subject}
    Implement a client and server for weather measurements and forecasts
  \end{block}

  \begin{block}{Motivation}
    \begin{itemize}
      \item practical experience with spatial databases like postgres/postgis
      \item model data in raster and vector representation
      \item visualize spatial data on a dynamic map
    \end{itemize}
  \end{block}
\end{frame}

\section{Data sources}
\subsection{Measurments}
\begin{frame}{Data sources - Measurements}
  \begin{block}{\textit{Deutsche Wetterdienst} weather stations}
    \begin{itemize}
        \item 503 weather stations in germany
        \item measurements like temperature, air pressure and so on.
        \item data available through public ftp server \\
          \vspace{0.1cm}
          \url{ftp://ftp.dwd.de/pub/CDC/observations_germany/}
        \item data for the past (several month up to years) until today
    \end{itemize}
  \end{block}
\end{frame}

\begin{frame}{Data sources - Measurements}
  \begin{block}{Approach}
    \begin{itemize}
        \item download stations metadata and measurements \\
          \vspace{0.1cm}
          \textbf{Problem:} station measures for a point (not region)
        \item use irregular tesselation (voronoi) to calculate region
    \end{itemize}
  \end{block}
\end{frame}

\begin{frame}{Data sources - Measurements}
  \headline{DWD weather stations}
  \begin{figure}
    \centering
    \includegraphics[width=0.8\textwidth]{images/stations.png}
    \caption{weather stations of Deutsche Wetterdienst}
    \label{fig:stations}
  \end{figure}
\end{frame}

\begin{frame}{Data sources - Measurements}
  \headline{Natural Earth \small{Germany}}
  \begin{figure}
    \centering
    \includegraphics[width=0.8\textwidth]{images/stations_in_germany.png}
    \caption{weather stations on top of polygon of germany}
    \label{fig:germany}
  \end{figure}
\end{frame}

\begin{frame}{Data sources - Measurements}
  \headline{Voronoi}
  \begin{figure}
    \centering
    \includegraphics[width=0.8\textwidth]{images/voronoi.png}
    \caption{germany divided into voronoi cells based on weather stations}
    \label{fig:voronoi}
  \end{figure}
\end{frame}

\subsection{Forecast}
\begin{frame}{Data sources - Forecast}
  \begin{block}{NOAA Global forecast system}
    \begin{itemize}
      \item global weather forecast model
      \item data public available for current and past forecasts
      \item data format grib2 (raster)
      \end{itemize}
  \end{block}
\end{frame}

\begin{frame}{Data sources - Forecast}
  \headline{Raster data}
  \begin{figure}
    \centering
    \includegraphics[width=0.8\textwidth]{images/raster.png}
    \caption{24h forecast for \textit{germany} 2015-02-05 18:00}
    \label{fig:voronoi}
  \end{figure}
\end{frame}

\begin{frame}{Data sources - Forecast}
  \headline{Raster data}
  \begin{figure}
    \centering
    \includegraphics[width=0.8\textwidth]{images/raster_in_germany.png}
    \caption{24h forecast for \textit{germany} 2015-02-05 18:00}
    \label{fig:voronoi}
  \end{figure}
\end{frame}

\begin{frame}{Data sources - Forecast}
  \headline{Raster data \small{resized and resampled (cubic interpolation)}}
  \begin{figure}
    \centering
    \includegraphics[width=0.8\textwidth]{images/raster_in_germany_resampled.png}
    \caption{24h forecast for \textit{germany} 2015-02-05 18:00}
    \label{fig:voronoi}
  \end{figure}
\end{frame}

\section{Implementation}
\subsection{Schema}
\begin{frame}{Implementation - Schema}
  \begin{block}{Measurement}
    \begin{itemize}
      \item station models weather station
      \item a station measures arbitrary data (measurements)
    \end{itemize}
  \end{block}

  \begin{figure}
    \centering
    \includegraphics[width=0.4\textwidth]{images/schema_station_measurement.png}
  \end{figure}
\end{frame}

\begin{frame}{Implementation - Schema}
  \begin{block}{Forecast}
    \begin{description}
      \item [date] computation date
      \item [interval] time interval in future
      \item [rast] forecast data in raster format (on several bands)
    \end{description}
  \end{block}

  \begin{figure}
    \centering
    \includegraphics[width=0.2\textwidth]{images/schema_forecast.png}
  \end{figure}
\end{frame}

\subsection{Server}
\begin{frame}{Implementation - Server}
  \begin{block}{Overview}
    \begin{description}
      \item [language] python
      \item [database] postgres+postgis
      \item [architecture] REST api
    \end{description}
  \end{block}

  \begin{block}{Libraries}
    \begin{itemize}
      \item flask web framework
      \item geoalchemy
      \item numpy
      \item shapely
    \end{itemize}
  \end{block}
\end{frame}

\subsection{Client}
\begin{frame}{Implementation - Client}
  \begin{block}{Overview}
    \begin{description}
      \item [language] javascript
      \item [visualization] html/svg
    \end{description}
  \end{block}

  \begin{block}{Libraries}
    \begin{itemize}
      \item leaflet
      \item jquery
      \item spin
    \end{itemize}
  \end{block}
\end{frame}

\section{Presentation}
\begin{frame}{Presentation}
  \center{\Huge{Demo}}
\end{frame}

\begin{frame}{Presentation - Measurements}
  \begin{figure}
    \centering
    \includegraphics[width=0.8\textwidth]{images/live_measurement.png}
    \caption{weather for germany 2012-02-08 00:00}
    \label{fig:voronoi}
  \end{figure}
\end{frame}

\begin{frame}{Presentation - Forecast}
  \begin{figure}
    \centering
    \includegraphics[width=0.8\textwidth]{images/live_forecast.png}
    \caption{forecast for 2015-02-09 00:00 for germany 24h before}
    \label{fig:voronoi}
  \end{figure}
\end{frame}

\section{Outlook and summary}
\begin{frame}{Outlook}
		\begin{block}{Possible extensions}
			\begin{description}
				\item [calculations] creation of own forecast system
				\item [datasets] include and integrate more forecast sources
				\item [visualisation] further features of a weather client
			\end{description}
		\end{block}
\end{frame}

\begin{frame}{Summary}
	\begin{block}{lessons from a mini project}
			\begin{itemize}
				\item learned to account for vector and raster data in postgis
				\item learned latest visualisations frameworks in python
			\end{itemize}
	\end{block}
\end{frame}

\begin{frame}{End}
  \center{\Huge{Question?}}\\
  \center{\huge{Feedback?}}\\
\end{frame}

\end{document}
